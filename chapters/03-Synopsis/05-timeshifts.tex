%!TEX root = ../Thesis.tex
\section{Machine Learning in 4D Seismic Time-Shift Extraction}

This final chapter consists of the submitted journal paper \citetitle{dramsch20193dwarping} \citep{dramsch20193dwarping}. This paper presents a novel 3D warping technique for the estimation of 4D seismic time-shifts. The algorithm is unsupervised and provides 3D warp-fields with uncertainty measures, while avoiding many limiting assumptions.

4D seismic time shift extraction is often done in 1D, due to time constraints and often sub-par performance of 3D algorithms. This chapter explores and summarizes conventional 3D warping methods and machine learning approaches. Many of these algorithms rely on classical cross-correlational or optical flow approaches. Correlation-based algorithms can be susceptible to noise and inversion-based algorithms can take weeks to provide results and optical flow approaches suffer from the implicit assumption in standard implementations. These approaches suffer from the same limitations in \acl{ml} systems just like conventional algorithms. In this chapter the medical Voxelmorph algorithm is adapted to match 4D seismic data volumes in 3D.

The Voxelmorph algorithm is based on the diffeomorphic assumption, which at its core describes the map of one data set to another data set, providing this map with particular properties. The main benefit of applying diffeomorphic mapping to geoscience data comes in the fact that all diffeomorphisms are homeomorphic. The homeomorphic assumption transfers well to the geological reality that the mathematical topology stays constant, resulting in reflectors neither crossing nor generating loops. 

The algorithm is trained in an unsupervised, or rather self-supervised way to avoid the bias from time shifts that were extracted from any other method. Supervised training is discussed in the paper as implicitly introducing the assumption of the extraction method for the training data into the newly trained network.

\begin{figure}
    \centering
    \includegraphics[width=\textwidth]{figures/Voxelmorph.pdf}
    \caption{Voxelmorph Architecture 2D abstraction. Two 3D volumes are passed to the network, concatenated (purple) and passed to a U-Net architecture. The U-Net outputs two cubes that generate the mean static velocity and the standard deviation, which is sampled during training. The sampled value is integrated to obtain the diffeomorphic warp velocity used in the spatial transformer layer (green). The network evaluates losses on the \ac{kl}-divergence at $\mu, \sigma$ and \ac{mse} between the warp result of the monitor volume and the warped base volume, enabling self-supervised training \citep[from][]{dramsch20193dwarping}}
    \label{fig:voxelmorph}
\end{figure}

The architecture in the network uses the U-Net architecture to input two 3D seismic volumes and extract a static warp velocity field, shown in \cref{fig:voxelmorph}. The static velocity field is extracted as a Gaussian distribution to measure the co-variance and provide uncertainty value of the three-dimensional warp field. The neural network itself does not warp the seismic data, to increase transparency of the process. The architecture following the U-net samples the extracted velocity distribution and integrates this value to obtain the diffeomorphic flow. These values are passed to a dense 3D warping mechanism to enable the unsupervised training. The losses involved are a \acf{kl}-divergence on the stationary velocity field and \ac{mse} on the difference between that warped monitor volume and the base volume.

In the paper we present the modified self-supervised \acl{nn} system and test the results on the training data itself and two generalization test sets. The first test set is on the same field but recorded at different times to the training set, ensuring a similar underlying geology, whereas, the second test set is taken from an adjacent field, recorded at different times, testing the full transfer of the trained network. We go on to test the original Voxelmorph architecture, which uses upsampled velocity fields and evaluate the results against our modified architecture, which uses the full flow field. Overall, this technique introduces a generalizable \acl{dl} approach to extract 3D time-shifts with uncertainty measures from raw stacked 4D seismic data.