%!TEX root = ../Thesis.tex
\chapter{Abstract}
Machine Learning provides an important tool for the modelling and analysis of geoscientific data. I have placed recent developments in deep learning into the greater context of machine learning by prefacing my work with a comprehensive history of machine learning and have reviewed the approaches and challenges of the use of machine learning in geoscience specifically. I have compiled a synopsis of the following chapters that contain topical peer-reviewed papers. 

This thesis follows the full data science workflow, starting with familiarization of the data domain in one published journal article, one published conference paper and one published workshop paper. It is followed by groundwork on machine learning and data processing on 4D seismic data in two submitted journal articles, one published conference paper and one published workshop paper. Based on this groundwork, I present a method for 4D seismic inversion in two published workshop papers. Finally, I present a novel unsupervised 3D time-shift extraction method for 4D seismic in one submitted journal paper. 

The aim of this thesis is to apply recent developments in computer vision systems and neural networks to physical data, particularly 4D seismic analysis. Neural networks are a type of machine learning that has made significant contributions to modern artificial intelligence and automatization. The applicability of neural networks for their capability of being a universal function approximator was recognized within geophysics from an early stage. With the deep learning boom, neural networks have experienced a renaissance in geoscience applications, particularly automatic seismic interpretation, inversion processes and sequence modelling.

The data for this thesis was acquired in the Danish North Sea, which contains chalk deposits, a sedimentologically distinct feature in the seismic data. The hydrocarbon-reservoir within the chalk has been subject to well-log analysis and core sampling in addition to seismic interpretation and 4D seismic analysis. During familiarization with the data, a new method to delineate chalk sediment in back-scatter scanning electron microscopy is introduced. Moreover, core fracture patterns, well imaging and seismic data are analyzed and compiled into a new workflow to ensure alignment of local and regional stress regimes.

Considering the wide interest in machine learning, my research investigates the following assumptions. The first paper shows that using pre-trained neural networks on natural images can reduce the data necessary for transfer learning to geoscience problems. I go on to analyze aliasing in neural networks and built a framework for complex-valued convolutional and dense neural networks to test the assumption that phase information can be implicitly learnt by real-valued neural networks. I further show that complex-valued convolutions can stabilize training and data compression on non-stationary physical data.

During the external research stay, a collaboration with an expert on Bayesian inversion for pressure-saturation inversion from 4D seismic amplitude difference maps resulted in a novel deep dense sample-based encoder-decoder network that learns the inversion process. The network contains a low-assumption physical basis (AVO) and learns the residual for the inversion process. My work shows that transfer from simulation data to field data is possible.

Finally, an unsupervised method is devised to extract 3D time-shifts from two 4D seismic cubes. The network extracts these 3D time-shifts with the inclusion of uncertainty measures. Commonly, time-shifts are extracted in 1D, due to processing speed, computational cost and poor performance of 3D methods. Within the training loop, the stationary velocity field is numerically integrated to obtain a diffeomorphic warp field that constrains the topology in a geologically consistent manner. The unsupervised implementation of the network structure ensures that biases from other time-shift extraction methods are not implicitly included in the network.

Overall, this thesis presents two new methods for the application of deep learning in 4D seismic analysis. Moreover, this thesis dives into information-theoretical implications of neural networks for non-stationary data such as seismic, and presents several ways to apply deep learning in a data regime, where ground truth is expensive, sparse, and sometimes impossible to obtain. These include transfer learning of pre-trained networks and transfer from simulation to field data. Additionally, we show an application of unsupervised learning, by devising a way of behaviour for the network to follow instead of supplying ground truth labels. Moreover, this results in a way to increase trust in the system, by limiting the extraction process to the deep learning system and performing well-defined operations within the network to automate the training, therefore, making the process transparent.


\chapter{Dansk Resum\'e}
Maskinl{\ae}ring (’machine learning’) er et vigtigt redskab til modellering og analyse af geovidenskabelige data. Jeg har sat den seneste udvikling inden for dyb l{\ae}ring (’deep learning’) ind i en st{\o}rre sammenh{\ae}ng via et forord, der gennemg{\aa}r maskinl{\ae}ringens historie, samt metoderne til og udfordringerne ved brug af maskinl{\ae}ring, specifikt inden for geovidenskab. Afhandlingen best{\aa}r af en synopsis af de publicerede og indsendte peer-reviewed afhandlinger i de derefter f{\o}lgende kapitler. 

Denne afhandling starter med at give overblik over af data via en publiceret tidsskriftsartikel, en publiceret konferenceartikel og en publiceret workshopartikel. Dern{\ae}st f{\o}lger forarbejdet til maskinl{\ae}ring og databehandling af 4D seismiske data i to indsendte tidsskriftsartikler, en publiceret konferenceartikel og en publiceret workshopartikel. P{\aa} baggrund af dette forarbejde, pr{\ae}senterer jeg en metode til 4D seismisk inversion i to publicerede workshopartikler. Endelig pr{\ae}senterer jeg en ny, 3D ekstraktionsmetode med tidsforskydning (’time shift’) for 4D seismiske data med brug af unsupervised learning i en indsendt tidsskriftartikel. 

Form{\aa}let med denne afhandling er, at anvende den seneste udvikling inden for systemer for computer vision og neurale netv{\ae}rk for fysiske data, is{\ae}r 4D seismisk analyse. Neurale netv{\ae}rk er en type maskinl{\ae}ring, der har bidraget stort inden for moderne, kunstig intelligens og automatisering. Det blev p{\aa} et tidligt tidspunkt anerkendt inden for geofysik, at neurale netv{\ae}rk var anvendelige. Med fremgangen inden for dyb l{\ae}ring, har neurale netv{\ae}rk oplevet en ren{\ae}ssance inden for geovidenskabelige anvendelser, is{\ae}r automatisk seismisk tolkning, inversionsprocesser og sekvensmodellering.

Data til denne afhandling, er indhentet fra den danske del af Nords{\o}en, der indeholder kridtaflejringer, hvilket er et sedimentologisk distinkt udtryk i de seismiske data. Kulbrinte-reservoiret i kridtet har v{\ae}ret genstand for borehulsanalyse og kernepr{\o}vetagning samt seismisk tolkning og 4D seismisk analyse. I forbindelse med arbejdet blev en ny metode til afgr{\ae}nsning af kridtsediment i scanning-elektronmikroskopi med tilbagespredning (’BSEM’) introduceret. Desuden blev kernefrakturm{\o}nstre, borehulsbilleder og seismiske data analyseret og samlet i en ny arbejdsgang for at justere lokale- og regionale stressregimer.

Set i lyset af den store interesse for maskinl{\ae}ring, unders{\o}ger min forskning flere omr{\aa}der. Den f{\o}rste artikel viser, at brugen af tr{\ae}nede neurale netv{\ae}rk p{\aa} billeder, kan reducere de data, der er n{\o}dvendige for at overf{\o}re l{\ae}ring til geovidenskabelige problemer. Jeg forts{\ae}tter med at analysere aliasing i neurale netv{\ae}rk og udvikler et computerprogram til at bygge neurale netv{\ae}rk, som bruger komplekse tal. Jeg sammenligner dette med netv{\ae}rk, som kun bruger ikke-komplekse tal, for at teste gendannelse af data og datakomprimering af ikke-station{\ae}re fysiske data.

Under det eksterne forskningsophold kom et samarbejde i stand, der udarbejder et nyt dybt, t{\ae}t pr{\o}vebaseret indkoder-dekoder-netv{\ae}rk, der l{\ae}rer inversionsprocesser. Netv{\ae}rket indeholder et fysisk grundlag, for selv at l{\ae}re resten af inversionsprocessen. Mit arbejde viser, at overf{\o}rsel fra indhentet data til simulationsdata er muligt.

Endelig blev der udviklet en ’unsupervised’ metode, til at udregne 3D-tidsforskydninger fra to 4D seismiske kuber. P{\aa} grund af de beregningsm{\ae}ssige omkostninger og d{\aa}rlige kvalitet, bliver disse normalt kun beregnet i 1D. Det neurale netv{\ae}rket beregner 3D tidsforskydningerne inklusiv usikkerhedsm{\aa}linger og er brugt p{\aa} tre forskellige seismiske datas{\ae}t. Den ’unsupervised’ implementering af netv{\ae}rksstrukturen sikrer, at bias fra andre tidsforskydnings ekstraktionsmetoder ikke implicit indg{\aa}r i netv{\ae}rket.

Samlet set, pr{\ae}senterer denne afhandling nye metoder inden for anvendelsen af dyb l{\ae}ring i 4D seismisk analyse. Desuden ser afhandlingen p{\aa} de informationsteoretiske konsekvenser af neurale netv{\ae}rk til ikke-station{\ae}re data, s{\aa}som seismik, og pr{\ae}senterer flere m{\aa}der at anvende dyb l{\ae}ring p{\aa} i et dataregime, hvor faktiske data er dyre at indsamle, af d{\aa}rlig kvalitet og nogle gange umulige at fremskaffe. Disse inkluderer overf{\o}rsel af l{\ae}ring i pr{\ae}-tr{\ae}nede (’pre-trained’) netv{\ae}rk og overf{\o}rsel fra simulationsdata til m{\aa}lt/indsamlet data.