% 4D Seismic
\section{4D seismic}

4D seismic is the analysis of seismic data that was acquired over the same location after some calendar time has passed. The repeated imaging of the same subsurface location, highlights changes in the subsurface that can lead to improved understanding of subsurface processes and fluid movement. E\&P companies in particular have an interest in imaging hydrocarbon reservoirs \citep{Johnston2013-jg}, however 4D seismic imaging wide applications for subsurface characterization, such as observing volcanic activitiy \citep{londono20184d} or CO2 sequestration monitoring \citep{Arts2004-ym}. 

The main applications of 4D seismic analysis according to \citet{Yilmaz2003-hp,Johnston2013-qg} include:
\begin{itemize}
\item Tracking fluid movement (steam, gas, and water)
\item Monitoring pressure depletion and validating depletion plans
\item Fault property estimation i.e. sealing or leaking faults
\item Locating bypassed oil imaging in heterogeneous reservoirs
\item Validating and updating geological and reservoir-simulation models
\end{itemize}

4D seismic data analysis suffers from the superposition of multiple effects on the seismic imaging. These effects include changes in the acquisition equipment due to technological advances, changes in acquisition geometry (source-receiver mismatch), as well as physical changes in the subsurface \citet{Yilmaz2003-hp, Johnston2013-jg}. These physical changes are in part due to fluid movement in the subsurface \citep{lumley1995seismic}, as well as, changes in the geology due to compaction and expansion \citep{Hatchell2005-op}. These geomechanical effects change the position of the reflectors, the thickness of stratigraphy and the physical properties such as density and wave velocity \citep{Herwanger2015-qz}.

Succesfull 4D applications rely on careful acquisition planning, closely matching the mismatch of source ($\Delta S$) and receiver ($\Delta R$). This awareness has generally improved the repeatability of seismic acquisition, however, the \ac{nrms} remains to be an important measure of noise sources that deteriorate the 4D seismic analysis. Moreover, 4D seismic analysis has brought to light that some 3D seismic processing workflows are not as repeatable and amplitude-preserving as they were thought to be \citep{Lumley2001-kx}. Modern processing flows include co-processing of the base and monitor seismic volumes with specialized tools to reduce differences from processing \citep{Johnston2013-qg}.

% Amplitude Differencing
% Time Shift Analysis
The standard analysis tool in 4D seismic interpretation are amplitude differences \citep{Johnston2013-jg}. Differences can stem from fluid movement or replacement and changes in the rock matrix due to compaction or temperature changes. Additionally, by-passed oil zones in heterogeneous reservoirs can be identified by "low difference zones" in generally mobile reflector packets \citep{Yilmaz2003-hp}. Usually, a simple difference of the 3D seismic volumes will not yield satisfactory results due to small-scale fluctuations in both arrival times and amplitudes, making time-shift analysis an important process to match the reflection events. These time-shift values have been shown to be a valuable source of information themselves \citep{Hall2002-dt,Hatchell2005-eg}, considering their sole dependence on wavefield kinematics, time shifts tend to be a more robust measurement than amplitude differences \citep{Johnston2013-jg}.

Considering normal incidence on a horizontal layer of thickness $z$ and a P-wave velocity $v$ with a traveltime $t$, we can express the changes in traveltime as:

\begin{equation}
    \frac{\Delta t}{t} = \frac{\Delta z}{z} - \frac{\Delta v}{v},
    \label{eq:timestrain}
\end{equation}

for homogeneous isotropic $v$ and small changes in $z$ and $v$. Originally developed in \citet{Hatchell2005-eg}, with a rigorous integral derivation presented in \citet{macbeth2019post}.

The vertical strain $\frac{\Delta z}{z}$ directly relates to the geomechanical strain $\xi_{zz}$, describing the vertical strain on the vertical surface of a infinitesemal element \citep{Herwanger2015-qz}. Independently \citet{Hatchell2005-eg} and \citet{roste2006estimation} developed a single-parameter solution to relate velocity changes and vertical strain

\begin{equation}
    \frac{\Delta v}{v} = - R \xi_{zz}
    \label{eq:R}
\end{equation}
with $R$ being the single parameter \ac{hbr}-factor \citep{Hatchell2005-op, macbeth2019post}. The \ac{hbr} being a lithological constant, we can relate \cref{eq:R} and \cref{eq:timestrain} and obtain a direct relationship between the vertical strain $\xi_{zz}$ and the time shift $\Delta t$ for a given lithology with property $R$

\begin{equation}
    \Delta t = t \cdot (1 + R) \cdot \xi_{zz}.
\end{equation}

Contingent on the assumption of zero-offset incidence, homogeneous velocity and isotropy, time shift extraction is mostly performed in z-direction by comparing traces directly. Prominently, the 1D windowed cross-correlation is used due to its computational speed and general lack of limiting underlying assumptions \citep{Rickett2001-nx}. The main drawback of this method is, however, that the result is highly dependent on the window-size and susceptible to noise. Other methods for post-stack seismic time shift extraction include \ac{dtw} \citep{Hale2013} and inversion-based approaches \citep{Rickett2007-yo}. 

More recently research into pre-stack time shift extraction and 3D-based methods is conducted. These methods relax the constraints of some assumptions of 1D applications \citep{ghaderi2005pre, hall2002time}. 3D time shifts have the ability to capture subsurface movement of reflectors and account for 3D effects of the $\Delta R / \Delta S$ acquisition mismatch, which effect seismic illumination.

\acf{qi} extends the interpretation of 4D changes to estimate fluid saturation and pressure changes within the reservoir. The subsurface changes recorded by the seismic data can be related to subsurface changes. These changes include fluid saturation and pressure changes, with the inversion process being non-unique and often reliant on prior information. The decoupling of pressure and saturation changes is non-trivial and relies on pre-stack or angle-stack information \citep{Landro2001-rz}.

Active areas of research in 4D seismic are the use of 4D seismic data to estimate saturation and pressure changes quantitatively particularly in volumetric applications as opposed to map-based approaches. However, these approaches often depend on reliable rock-physics models, an area of research in model-based approaches. Moreover, there's active research in moving to volumetric approaches in time-shift estimation and quantitative pre-stack analysis. Additional research in extractive data-based methods and model-based approaches investigate how much information is available directly from the data and what information is available from the modelling feedback-loop.