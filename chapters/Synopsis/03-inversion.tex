%!TEX root = ../Thesis.tex
\section{Machine Learning Application I}

% dramsch2019physics
% In this work we present a deep neural network inversion on map-based 4D seismic data for pressure and saturation. We present a novel neural network architecture that trains on synthetic data and provides insights into observed field seismic. The network explicitly includes AVO gradient calculation within the network as physical knowledge to stabilize pressure and saturation changes separation. We apply the method to Schiehallion field data and go on to compare the results to Bayesian inversion results. Despite not using convolutional neural networks for spatial information, we produce maps with good signal to noise ratio and coherency.

% dramsch2019deep
% Geoscience data often have to rely on strong priors in the face of uncertainty. Additionally, we often try to detect or model anomalous sparse data that can appear as an outlier in machine learning models. These are classic examples of imbalanced learning. Approaching these problems can benefit from including prior information from physics models or transforming data to a beneficial domain. We show an example of including physical information in the architecture of a neural network as prior information. We go on to present noise injection at training time to successfully transfer the network from synthetic data to field data.
