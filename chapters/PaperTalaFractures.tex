%!TEX root = ../Thesis.tex
%\chapter{Long chapter title with $\pi$ $π$ or π}
%\chapter{Long chapter title with \texorpdfstring{$\pi$ $π$ or π}{π π or π}}
\chapter{An Integrated Approach to Fracture Characterization of the Kraka
Field}

Oil and gas production of tight chalk reservoirs frequently rely on the presence of natural fractures, whichincreases the effective permeability of the reservoirs.  Knowledge of these fracture systems can therefore beused strategically in well planning as well as in IOR and EOR efforts.  Here we present an integrated workflowfor fracture characterization in chalk, developed in the Kraka Field, located in the Danish sector of the NorthSea.  The workflow is based on data from borehole images, cores and seismic.  By introducing two ant-trackedattribute volumes, which display structural trends below the resolution of amplitude seismic, we are able tocorrelate features at different scales.  In Kraka, this approach has revealed that the fracture pattern is morecomplex than previously suggested.  We propose that fracture generation and propagation in the field is inpart controlled by the regional maximum horizontal stress and in part formed in response to salt movements.

{\vfill\hfill\newline\fbox{\parbox{.97\textwidth}{\fullcite{aabo2018integrated}}}}

\includepdf[pages=5-20,pagecommand={},width=1.2\textwidth,offset=0.7cm 0.1cm]{papers/2018.1}
\todo{Check Page Numbers}
\todo{Fix Alignment}

