% 4D Seismic
\todo{Write 4D seismic intro}
\section{4D seismic}

\todo{Motivation of doing 4D seismic}

4D seismic is the analysis of seismic data that was acquired over the same area after some calendar time has passed. This analysis usually includes matching two seismic cubes and analyzing the shifts as well as the amplitude difference after alignment. These can theoretically be inverted for physical properties that cause the change of the seismic image.

4D seismic data analysis suffers from the superposition of multiple effects on the seismic imaging. These effects include changes in the acquisition equipment due to technological advances, changes in acquisition geometry (source-receiver mismatch), as well as physical changes in the subsurface. These physical changes are in part due to fluid movement in the subsurface, as well as, changes in the geology due to compaction and expansion. These geomechanical effects change the position of the reflectors, the thickness of stratigraphy and the physical properties such as density and wave velocity.
% Amplitude Differencing

\todo{Write amp diff intro}

Amplitude differences are the standard analysis tool in 4D sesimic interpretation. Once the seismic cubes have been aligned, the amplitude difference can be interpreted by experts. Interpreters will look for both the differences to see fluid movements. Additionally, by-passed zones can be identified by "low difference zones" in generally mobile reflector packets.
% Time Shift Analysis
\todo{Write time shift intro}

Time shifts were a main tool to align seismic data for amplitude difference interpretation. The time shifts and time strains themselves can be interpreted on their own. Time shifts extraction is mostly done in z-direction by comparing traces. The most common methods implementing time shift extraction operate solely in 1D on traces. Prominently, the 1D windowed cross-correlation is used due to its computational speed and general robustness. The main drawback of this method is, however, that the result is highly dependent on the window-size. 
% Methods
% Cross Correlation
% Taylor Expansion Methods
% Optical Flow Method

More recently research into pre-stack time shift extraction and 3D-based methods is conducted. These methods take into account changes in the geology and seismic illumination.
% 3d methods
% Inversion

\todo{Write inversion bit intro}
The subsurface changes recorded by the seismic data can be related to subsurface changes. These changes include porosity, fluid saturation and pressure changes. This inversion process is non-unique and is often reliant on prior information. Many inversions rely on Bayesian processes to de-risk the inversion. 
