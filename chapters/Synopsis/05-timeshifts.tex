%!TEX root = ../Thesis.tex
\section{Machine Learning in 4D Seismic Time-Shift Extraction}

This final chapter consists of the submitted journal paper \citetitle{dramsch20193dwarping}. This paper presents a novel 3D warping technique for the estimation of 4D seismic time-shifts.

4D seismic time shift extraction is often done in 1D, due to time constraints and often sub-par performance of 3D algorithms. This chapter explores and summarizes conventional 3D warping methods and machine learning approaches. Many of these algorithms rely on classical cross-correlational or optical flow approaches. These approaches suffer from the same limitations in \acl{ml} systems just like conventional algorithms. In this chapter I apply the medical Voxelmorph algorithm to match 4D seismic data.

The main assumption in the Voxelmorph image alignment depends on diffeomorphic warp fields. The diffeomorphic assumption transfer well to the geological reality that the the mathematical topology stays constant, resulting in reflectors neither crossing nor generating loops. 




% dramsch20193dwarping

% We present a novel 3D warping technique for the estimation of 4D seismic time-shift. This unsupervised method provides a diffeomorphic 3D time shift field that includes uncertainties, therefore it does not need prior time-shift data to be trained. This results in a widely applicable method in time-lapse seismic data analysis. We explore the generalization of the method to unseen data both in the same geological setting and in a different field, where the generalization error stays constant and within an acceptable range across test cases. We further explore upsampling of the warp field from a smaller network to decrease computational cost and see some deterioration of the warp field quality as a result.