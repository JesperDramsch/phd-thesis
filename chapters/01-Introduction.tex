%!TEX root = ../Thesis.tex
\chapter{Introduction}

This thesis explores machine learning in geoscience with a special focus on deep learning in 4D seismics. Recently, machine learning and neural networks in particular have made important impacts in many scientific disciplines, with geoscience exploring these new approaches as well. This study contributes to this body of emerging work in deep neural networks and computer vision systems for the modelling and analysis of geoscientific data. The main contribution being a physics-based neural architecture for pressure-saturation inversion and a novel algorithm for 3D time-shift extraction in 4D seismic.

The growing interest in machine learning sometimes overlooks the fact that machine learning as a concept was introduced in 1950. Geoscience and in particular geophysics has followed the innovation in artificial intelligence and especially neural networks closely. Early applications of neural networks include seismic processing and seismic inversion. Moreover, Gaussian processes were early introduced in geostatistics as kriging, being the primary application of Gaussian processes for a period of time. Deep learning becoming popular and particularly breakthroughs in computer vision have sparked interest in applying machine learning computer vision to seismic interpretation in the hopes for increased accuracy, reproducibility and automation.

In recent years, 4D seismic itself has made an impact in geoscience. The method enables imaging of changes in the subsurface. This is essential in hydrocarbon production, enabling extended production reducing the direct environmental footprint and ensuring resource safety. Moreover, it enables CO\textsubscript{2} sequestration monitoring for reservoir and seal integrity and applications including nuclear test treaty compliance, waste storage, and deep geothermal monitoring. 4D seismic matching has exposed deficits in 3D seismic processing, therefore furthered our understanding of amplitude-preserving and surface-consistent processing steps. Additionally, furthering our understanding of in-situ validation of geomechanical concepts and update of heterogeneous subsurface models.

The structure of this study is composed of a theoretical introduction into 4D seismic principles, followed by a thorough overview of the development of machine learning in general to provide context for a review of machine learning in geoscience. This leads into a discussion of challenges for machine learning in geoscience. The theoretical foundation serves as the basis for ten publications that are arranged into four topical chapters. 

The first chapter comprised of three papers investigates image processing for improved seismic interpretation. This workflow was essential in analyzing the local to regional stress fields from fracture and fault expressions in 3D seismic data. Familiarization with the available data set proved to be valuable in the analysis of the 4D seismic data set and fine-tuning machine learning algorithms to the specific domain presented. Then unsupervised machine learning is applied to distinguish chalk sediments in back-scatter electron microscopy, providing a machine learning solution for a normally manual and tedious task.

The second chapter comprised of four papers investigates fundamentals of signal processing for 4D seismic and neural networks. In which different metrics and constraints for dynamic time warping are explored, introducing a novel constraint for warping traces, significantly improving the alignment of 4D seismic traces in the base and monitor volume. Then aliasing and the impact of including phase information in neural networks is investigated. For the purpose of this study open source software was translated to the modern Tensorflow framework to enable building complex-valued convolutional neural networks. This chapter concludes in investigating transfer learning of pre-trained neural networks on natural images applied to seismic data, introducing a method to apply deep learning in label sparse environments.

The third chapter comprised of two papers introduces a deep neural network architecture for 4D quantitative pressure-saturation inversion. The regression model implements a layer that computes basic physical knowledge within the network architecture to stabilize the network. The physical knowledge encoded in the layer is the AVO gradient between the input seismic data. This data is passed into an variational encoder-decoder architecture. In this work we show that this network can be trained on simulation data and transferred to field data by applying Gaussian noise to the noise-free simulation input data to condition the network to accept noisy inputs from field data.

The fourth chapter comprised of a single paper introduces a robust method for 3D time shift extraction in 4D data. Time shifts in 4D data are commonly extracted in 1D due to computational cost and often poor performance of 3D methods. This method uses a deep learning system to extract the mapping of two seismic volumes without supplying a-priori time shift data, training self-supervised. Moreover, the method limits the neural network to the extraction of the stationary warp velocity field but leaves the warping to a non-learning 3D interpolation to increase transparency of the method. Additionally, the method supplies uncertainty values for the warp velocity. Constraining the possible 3D time shifts is important to ensure sensible results for the time shifts, as well as, the aligned monitor seismic. This is ensured by implementing a geologically intuitive constraint on the warp-field, namely a diffeomorphic mapping, which prohibits crossing or looping of reflectors after warping. This learning-based method can be trained in advance, providing timely results on unseen data, which is essential in 4D seismic analysis.